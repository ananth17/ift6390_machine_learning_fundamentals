
% Default to the notebook output style

    


% Inherit from the specified cell style.




    
\documentclass{article}

    
    
    \usepackage{graphicx} % Used to insert images
    \usepackage{adjustbox} % Used to constrain images to a maximum size 
    \usepackage{color} % Allow colors to be defined
    \usepackage{enumerate} % Needed for markdown enumerations to work
    \usepackage{geometry} % Used to adjust the document margins
    \usepackage{amsmath} % Equations
    \usepackage{amssymb} % Equations
    \usepackage[mathletters]{ucs} % Extended unicode (utf-8) support
    \usepackage[utf8x]{inputenc} % Allow utf-8 characters in the tex document
    \usepackage{fancyvrb} % verbatim replacement that allows latex
    \usepackage{grffile} % extends the file name processing of package graphics 
                         % to support a larger range 
    % The hyperref package gives us a pdf with properly built
    % internal navigation ('pdf bookmarks' for the table of contents,
    % internal cross-reference links, web links for URLs, etc.)
    \usepackage{hyperref}
    \usepackage{longtable} % longtable support required by pandoc >1.10
    \usepackage{booktabs}  % table support for pandoc > 1.12.2
    

    
    
    \definecolor{orange}{cmyk}{0,0.4,0.8,0.2}
    \definecolor{darkorange}{rgb}{.71,0.21,0.01}
    \definecolor{darkgreen}{rgb}{.12,.54,.11}
    \definecolor{myteal}{rgb}{.26, .44, .56}
    \definecolor{gray}{gray}{0.45}
    \definecolor{lightgray}{gray}{.95}
    \definecolor{mediumgray}{gray}{.8}
    \definecolor{inputbackground}{rgb}{.95, .95, .85}
    \definecolor{outputbackground}{rgb}{.95, .95, .95}
    \definecolor{traceback}{rgb}{1, .95, .95}
    % ansi colors
    \definecolor{red}{rgb}{.6,0,0}
    \definecolor{green}{rgb}{0,.65,0}
    \definecolor{brown}{rgb}{0.6,0.6,0}
    \definecolor{blue}{rgb}{0,.145,.698}
    \definecolor{purple}{rgb}{.698,.145,.698}
    \definecolor{cyan}{rgb}{0,.698,.698}
    \definecolor{lightgray}{gray}{0.5}
    
    % bright ansi colors
    \definecolor{darkgray}{gray}{0.25}
    \definecolor{lightred}{rgb}{1.0,0.39,0.28}
    \definecolor{lightgreen}{rgb}{0.48,0.99,0.0}
    \definecolor{lightblue}{rgb}{0.53,0.81,0.92}
    \definecolor{lightpurple}{rgb}{0.87,0.63,0.87}
    \definecolor{lightcyan}{rgb}{0.5,1.0,0.83}
    
    % commands and environments needed by pandoc snippets
    % extracted from the output of `pandoc -s`
    \DefineVerbatimEnvironment{Highlighting}{Verbatim}{commandchars=\\\{\}}
    % Add ',fontsize=\small' for more characters per line
    \newenvironment{Shaded}{}{}
    \newcommand{\KeywordTok}[1]{\textcolor[rgb]{0.00,0.44,0.13}{\textbf{{#1}}}}
    \newcommand{\DataTypeTok}[1]{\textcolor[rgb]{0.56,0.13,0.00}{{#1}}}
    \newcommand{\DecValTok}[1]{\textcolor[rgb]{0.25,0.63,0.44}{{#1}}}
    \newcommand{\BaseNTok}[1]{\textcolor[rgb]{0.25,0.63,0.44}{{#1}}}
    \newcommand{\FloatTok}[1]{\textcolor[rgb]{0.25,0.63,0.44}{{#1}}}
    \newcommand{\CharTok}[1]{\textcolor[rgb]{0.25,0.44,0.63}{{#1}}}
    \newcommand{\StringTok}[1]{\textcolor[rgb]{0.25,0.44,0.63}{{#1}}}
    \newcommand{\CommentTok}[1]{\textcolor[rgb]{0.38,0.63,0.69}{\textit{{#1}}}}
    \newcommand{\OtherTok}[1]{\textcolor[rgb]{0.00,0.44,0.13}{{#1}}}
    \newcommand{\AlertTok}[1]{\textcolor[rgb]{1.00,0.00,0.00}{\textbf{{#1}}}}
    \newcommand{\FunctionTok}[1]{\textcolor[rgb]{0.02,0.16,0.49}{{#1}}}
    \newcommand{\RegionMarkerTok}[1]{{#1}}
    \newcommand{\ErrorTok}[1]{\textcolor[rgb]{1.00,0.00,0.00}{\textbf{{#1}}}}
    \newcommand{\NormalTok}[1]{{#1}}
    
    % Define a nice break command that doesn't care if a line doesn't already
    % exist.
    \def\br{\hspace*{\fill} \\* }
    % Math Jax compatability definitions
    \def\gt{>}
    \def\lt{<}
    % Document parameters
    \title{Homework2}
    
    
    

    % Pygments definitions
    
\makeatletter
\def\PY@reset{\let\PY@it=\relax \let\PY@bf=\relax%
    \let\PY@ul=\relax \let\PY@tc=\relax%
    \let\PY@bc=\relax \let\PY@ff=\relax}
\def\PY@tok#1{\csname PY@tok@#1\endcsname}
\def\PY@toks#1+{\ifx\relax#1\empty\else%
    \PY@tok{#1}\expandafter\PY@toks\fi}
\def\PY@do#1{\PY@bc{\PY@tc{\PY@ul{%
    \PY@it{\PY@bf{\PY@ff{#1}}}}}}}
\def\PY#1#2{\PY@reset\PY@toks#1+\relax+\PY@do{#2}}

\expandafter\def\csname PY@tok@gd\endcsname{\def\PY@tc##1{\textcolor[rgb]{0.63,0.00,0.00}{##1}}}
\expandafter\def\csname PY@tok@gu\endcsname{\let\PY@bf=\textbf\def\PY@tc##1{\textcolor[rgb]{0.50,0.00,0.50}{##1}}}
\expandafter\def\csname PY@tok@gt\endcsname{\def\PY@tc##1{\textcolor[rgb]{0.00,0.27,0.87}{##1}}}
\expandafter\def\csname PY@tok@gs\endcsname{\let\PY@bf=\textbf}
\expandafter\def\csname PY@tok@gr\endcsname{\def\PY@tc##1{\textcolor[rgb]{1.00,0.00,0.00}{##1}}}
\expandafter\def\csname PY@tok@cm\endcsname{\let\PY@it=\textit\def\PY@tc##1{\textcolor[rgb]{0.25,0.50,0.50}{##1}}}
\expandafter\def\csname PY@tok@vg\endcsname{\def\PY@tc##1{\textcolor[rgb]{0.10,0.09,0.49}{##1}}}
\expandafter\def\csname PY@tok@m\endcsname{\def\PY@tc##1{\textcolor[rgb]{0.40,0.40,0.40}{##1}}}
\expandafter\def\csname PY@tok@mh\endcsname{\def\PY@tc##1{\textcolor[rgb]{0.40,0.40,0.40}{##1}}}
\expandafter\def\csname PY@tok@go\endcsname{\def\PY@tc##1{\textcolor[rgb]{0.53,0.53,0.53}{##1}}}
\expandafter\def\csname PY@tok@ge\endcsname{\let\PY@it=\textit}
\expandafter\def\csname PY@tok@vc\endcsname{\def\PY@tc##1{\textcolor[rgb]{0.10,0.09,0.49}{##1}}}
\expandafter\def\csname PY@tok@il\endcsname{\def\PY@tc##1{\textcolor[rgb]{0.40,0.40,0.40}{##1}}}
\expandafter\def\csname PY@tok@cs\endcsname{\let\PY@it=\textit\def\PY@tc##1{\textcolor[rgb]{0.25,0.50,0.50}{##1}}}
\expandafter\def\csname PY@tok@cp\endcsname{\def\PY@tc##1{\textcolor[rgb]{0.74,0.48,0.00}{##1}}}
\expandafter\def\csname PY@tok@gi\endcsname{\def\PY@tc##1{\textcolor[rgb]{0.00,0.63,0.00}{##1}}}
\expandafter\def\csname PY@tok@gh\endcsname{\let\PY@bf=\textbf\def\PY@tc##1{\textcolor[rgb]{0.00,0.00,0.50}{##1}}}
\expandafter\def\csname PY@tok@ni\endcsname{\let\PY@bf=\textbf\def\PY@tc##1{\textcolor[rgb]{0.60,0.60,0.60}{##1}}}
\expandafter\def\csname PY@tok@nl\endcsname{\def\PY@tc##1{\textcolor[rgb]{0.63,0.63,0.00}{##1}}}
\expandafter\def\csname PY@tok@nn\endcsname{\let\PY@bf=\textbf\def\PY@tc##1{\textcolor[rgb]{0.00,0.00,1.00}{##1}}}
\expandafter\def\csname PY@tok@no\endcsname{\def\PY@tc##1{\textcolor[rgb]{0.53,0.00,0.00}{##1}}}
\expandafter\def\csname PY@tok@na\endcsname{\def\PY@tc##1{\textcolor[rgb]{0.49,0.56,0.16}{##1}}}
\expandafter\def\csname PY@tok@nb\endcsname{\def\PY@tc##1{\textcolor[rgb]{0.00,0.50,0.00}{##1}}}
\expandafter\def\csname PY@tok@nc\endcsname{\let\PY@bf=\textbf\def\PY@tc##1{\textcolor[rgb]{0.00,0.00,1.00}{##1}}}
\expandafter\def\csname PY@tok@nd\endcsname{\def\PY@tc##1{\textcolor[rgb]{0.67,0.13,1.00}{##1}}}
\expandafter\def\csname PY@tok@ne\endcsname{\let\PY@bf=\textbf\def\PY@tc##1{\textcolor[rgb]{0.82,0.25,0.23}{##1}}}
\expandafter\def\csname PY@tok@nf\endcsname{\def\PY@tc##1{\textcolor[rgb]{0.00,0.00,1.00}{##1}}}
\expandafter\def\csname PY@tok@si\endcsname{\let\PY@bf=\textbf\def\PY@tc##1{\textcolor[rgb]{0.73,0.40,0.53}{##1}}}
\expandafter\def\csname PY@tok@s2\endcsname{\def\PY@tc##1{\textcolor[rgb]{0.73,0.13,0.13}{##1}}}
\expandafter\def\csname PY@tok@vi\endcsname{\def\PY@tc##1{\textcolor[rgb]{0.10,0.09,0.49}{##1}}}
\expandafter\def\csname PY@tok@nt\endcsname{\let\PY@bf=\textbf\def\PY@tc##1{\textcolor[rgb]{0.00,0.50,0.00}{##1}}}
\expandafter\def\csname PY@tok@nv\endcsname{\def\PY@tc##1{\textcolor[rgb]{0.10,0.09,0.49}{##1}}}
\expandafter\def\csname PY@tok@s1\endcsname{\def\PY@tc##1{\textcolor[rgb]{0.73,0.13,0.13}{##1}}}
\expandafter\def\csname PY@tok@sh\endcsname{\def\PY@tc##1{\textcolor[rgb]{0.73,0.13,0.13}{##1}}}
\expandafter\def\csname PY@tok@sc\endcsname{\def\PY@tc##1{\textcolor[rgb]{0.73,0.13,0.13}{##1}}}
\expandafter\def\csname PY@tok@sx\endcsname{\def\PY@tc##1{\textcolor[rgb]{0.00,0.50,0.00}{##1}}}
\expandafter\def\csname PY@tok@bp\endcsname{\def\PY@tc##1{\textcolor[rgb]{0.00,0.50,0.00}{##1}}}
\expandafter\def\csname PY@tok@c1\endcsname{\let\PY@it=\textit\def\PY@tc##1{\textcolor[rgb]{0.25,0.50,0.50}{##1}}}
\expandafter\def\csname PY@tok@kc\endcsname{\let\PY@bf=\textbf\def\PY@tc##1{\textcolor[rgb]{0.00,0.50,0.00}{##1}}}
\expandafter\def\csname PY@tok@c\endcsname{\let\PY@it=\textit\def\PY@tc##1{\textcolor[rgb]{0.25,0.50,0.50}{##1}}}
\expandafter\def\csname PY@tok@mf\endcsname{\def\PY@tc##1{\textcolor[rgb]{0.40,0.40,0.40}{##1}}}
\expandafter\def\csname PY@tok@err\endcsname{\def\PY@bc##1{\setlength{\fboxsep}{0pt}\fcolorbox[rgb]{1.00,0.00,0.00}{1,1,1}{\strut ##1}}}
\expandafter\def\csname PY@tok@kd\endcsname{\let\PY@bf=\textbf\def\PY@tc##1{\textcolor[rgb]{0.00,0.50,0.00}{##1}}}
\expandafter\def\csname PY@tok@ss\endcsname{\def\PY@tc##1{\textcolor[rgb]{0.10,0.09,0.49}{##1}}}
\expandafter\def\csname PY@tok@sr\endcsname{\def\PY@tc##1{\textcolor[rgb]{0.73,0.40,0.53}{##1}}}
\expandafter\def\csname PY@tok@mo\endcsname{\def\PY@tc##1{\textcolor[rgb]{0.40,0.40,0.40}{##1}}}
\expandafter\def\csname PY@tok@kn\endcsname{\let\PY@bf=\textbf\def\PY@tc##1{\textcolor[rgb]{0.00,0.50,0.00}{##1}}}
\expandafter\def\csname PY@tok@mi\endcsname{\def\PY@tc##1{\textcolor[rgb]{0.40,0.40,0.40}{##1}}}
\expandafter\def\csname PY@tok@gp\endcsname{\let\PY@bf=\textbf\def\PY@tc##1{\textcolor[rgb]{0.00,0.00,0.50}{##1}}}
\expandafter\def\csname PY@tok@o\endcsname{\def\PY@tc##1{\textcolor[rgb]{0.40,0.40,0.40}{##1}}}
\expandafter\def\csname PY@tok@kr\endcsname{\let\PY@bf=\textbf\def\PY@tc##1{\textcolor[rgb]{0.00,0.50,0.00}{##1}}}
\expandafter\def\csname PY@tok@s\endcsname{\def\PY@tc##1{\textcolor[rgb]{0.73,0.13,0.13}{##1}}}
\expandafter\def\csname PY@tok@kp\endcsname{\def\PY@tc##1{\textcolor[rgb]{0.00,0.50,0.00}{##1}}}
\expandafter\def\csname PY@tok@w\endcsname{\def\PY@tc##1{\textcolor[rgb]{0.73,0.73,0.73}{##1}}}
\expandafter\def\csname PY@tok@kt\endcsname{\def\PY@tc##1{\textcolor[rgb]{0.69,0.00,0.25}{##1}}}
\expandafter\def\csname PY@tok@ow\endcsname{\let\PY@bf=\textbf\def\PY@tc##1{\textcolor[rgb]{0.67,0.13,1.00}{##1}}}
\expandafter\def\csname PY@tok@sb\endcsname{\def\PY@tc##1{\textcolor[rgb]{0.73,0.13,0.13}{##1}}}
\expandafter\def\csname PY@tok@k\endcsname{\let\PY@bf=\textbf\def\PY@tc##1{\textcolor[rgb]{0.00,0.50,0.00}{##1}}}
\expandafter\def\csname PY@tok@se\endcsname{\let\PY@bf=\textbf\def\PY@tc##1{\textcolor[rgb]{0.73,0.40,0.13}{##1}}}
\expandafter\def\csname PY@tok@sd\endcsname{\let\PY@it=\textit\def\PY@tc##1{\textcolor[rgb]{0.73,0.13,0.13}{##1}}}

\def\PYZbs{\char`\\}
\def\PYZus{\char`\_}
\def\PYZob{\char`\{}
\def\PYZcb{\char`\}}
\def\PYZca{\char`\^}
\def\PYZam{\char`\&}
\def\PYZlt{\char`\<}
\def\PYZgt{\char`\>}
\def\PYZsh{\char`\#}
\def\PYZpc{\char`\%}
\def\PYZdl{\char`\$}
\def\PYZhy{\char`\-}
\def\PYZsq{\char`\'}
\def\PYZdq{\char`\"}
\def\PYZti{\char`\~}
% for compatibility with earlier versions
\def\PYZat{@}
\def\PYZlb{[}
\def\PYZrb{]}
\makeatother


    % Exact colors from NB
    \definecolor{incolor}{rgb}{0.0, 0.0, 0.5}
    \definecolor{outcolor}{rgb}{0.545, 0.0, 0.0}



    
    % Prevent overflowing lines due to hard-to-break entities
    \sloppy 
    % Setup hyperref package
    \hypersetup{
      breaklinks=true,  % so long urls are correctly broken across lines
      colorlinks=true,
      urlcolor=blue,
      linkcolor=darkorange,
      citecolor=darkgreen,
      }
    % Slightly bigger margins than the latex defaults
    
    \geometry{verbose,tmargin=1in,bmargin=1in,lmargin=1in,rmargin=1in}
    
    

    \begin{document}
    
    
    \maketitle
    
    

    
    \subsection{IFT6390}\label{ift6390}

\subsection{Homework 2: multilayer perceptron (single hidden
layer)}\label{homework-2-multilayer-perceptron-single-hidden-layer}

\subsubsection{Gabriel C-Parent C5912}\label{gabriel-c-parent-c5912}

    \subsection{Q1 a)}\label{q1-a}

    \textbf{What is the dimension of $b^{(1)}$ ?}

$b^{(1)}$ is $d_h \times 1$.

    \textbf{Write down the formula to calculate the vector of activations}
\textbf{(i.e.~before the non-linearity) of the neurons in the hidden
layer,} \textbf{$h^a$ , given an input, x, at first in matrix
expression.}

$h^a_{d_h \times 1} = b^{(1)}_{d_h \times 1} + W^{(1)}_{d_h \times d} x_{d \times 1} $

    \textbf{Element-by-element computation of the entries of $h^a$.}

$h^a_i = b^{(1)}_i + \sum\limits_{j=1}^d w^{(1)}_{i, j}x_j$

    \textbf{Write down the vector of outputs of the hidden layer neurons,
$h^s$, in terms of the activations, $h^a$.}

$h^s_{d_h \times 1} = tanh(b^{(1)}_{d_h \times 1} + W^{(1)}_{d_h \times d} x_{d \times 1}) $

    \subsection{Q1 b)}\label{q1-b}

    \textbf{Let $W^{(2)}$ be the weight matrix from the hidden to output
layer and $b^{(2)}$ be the} \textbf{vector of biases for the output
layer. What are the dimensions of $W^{(2)}$ et $b^{(2)}$ ?}

$b^{(2)}$ is $m \times 1$

$W^{(2)}$ is $m \times d_h$

    \textbf{Write down the formula describing the vector of activations of
neurons in the output} \textbf{layer $o^a$ given $h^s$ in matrix form}

$o^a_{m \times 1} = b^{(2)}_{m \times 1} + W^{(2)}_{m \times d_h} h^s_{d_h \times 1}$

    \textbf{Then in detail element-wise form}

$o^a_i = b^{(2)}_i + \sum\limits_{j=1}^{d_h} w^{(2)}_{i, j} h^s_j$

    \subsection{Q1 c)}\label{q1-c}

    \textbf{What is contained in the set of all network parameters,
$\theta$}

\begin{itemize}
\itemsep1pt\parskip0pt\parsep0pt
\item
  activation function (tanh, sigmoid, linear even)
\item
  number of hidden layer nodes
\item
  $W^{(1)}, W^{(2)}, b^{(1)}, b^{(2)}$
\end{itemize}

    \textbf{What is the number $n_{\theta}$ of parameters in $\theta$ ?}

\begin{itemize}
\itemsep1pt\parskip0pt\parsep0pt
\item
  $W^{(1)}$ is $d_h \times d$
\item
  $W^{(2)}$ is $m \times d_h$
\item
  $b^{(1)}$ is $d_h \times 1$
\item
  $b^{(2)}$ is $m \times 1$
\end{itemize}

$n_{\theta}$ = $d_h * (d+1 + m) + m$

    \subsection{Q1 d)}\label{q1-d}

    \textbf{Show that the gradients wrt. parameters $W^{(2)}$ and $b^{(2)}$
are given by:}

$\dfrac {\delta  L} {\delta W^{(2)}} = \dfrac {\delta L} {\delta o^a} (h^s)^T $

and

$\dfrac {\delta  L} {\delta b^{(2)}} = \dfrac {\delta L} {\delta o^a} $

    ** (i) the dimensions**

$\dfrac {\delta  L} {\delta W^{(2)}}$ is $m \times d_h$

$\dfrac {\delta L} {\delta o^a}$ is $m \times 1$

$(h^s)^T$ is $1 \times d_h$

$\dfrac {\delta  L} {\delta W^{(2)}}$ is $m \times 1$

\textbf{(ii) the weights}

$o^s = softmax(o^a) = softmax(W^{(2)}h^s + b^{(2)})$

$f(g(x))' = f'(g(x)) * g'(x)$

$\dfrac {\delta  L} {\delta W^{(2)}} = \dfrac {\delta L} {\delta o^a} * \dfrac {\delta (W^{(2)}h^s + b^{(2)})} {\delta W^{(2)}} = \dfrac {\delta L} {\delta o^a} (h^s)^T $

\textbf{(iii) the biases}

same as for the weights

$\dfrac {\delta  L} {\delta b^{(2)}} = \dfrac {\delta L} {\delta o^a} * \dfrac {\delta (W^{(2)}h^s + b^{(2)})} {\delta b^{(2)}} = \dfrac {\delta L} {\delta o^a} $

    \subsection{Q1 e)}\label{q1-e}

    Using the chain rule

\[ \dfrac {\delta L} {\delta h^s_j} = \sum \limits_{k=1}^M \dfrac {\delta L} {\delta o^a_k} \dfrac {\delta o^a_k} {\delta h^s_j}\]

show that the partial derivatives of the cost L with respect to the
outputs of the neurons in the hidden layer are given by:

\[\dfrac {\delta L} {\delta h^s_j} = (W^{(2)})^T \dfrac {\delta L} {\delta o^a}\]

We start from:

\[ \dfrac {\delta L} {\delta h^s_j} = \sum \limits_{k=1}^M \dfrac {\delta L} {\delta o^a_k} \dfrac {\delta o^a_k} {\delta h^s_j}\]

\[ o^a_k = W^{(2)}_k h^s_j + b^{(2)}_k\]

\[ \dfrac {\delta o^a_k} {\delta h^s_j} = W^{(2)}_k \]

We substitute the derivative term for its value

\[ \dfrac {\delta L} {\delta h^s_j} = \sum \limits_{k=1}^M \dfrac {\delta L} {\delta o^a_k}  W^{(2)}_k\]

Which is equivalent in the matrix form to

\[\dfrac {\delta L} {\delta h^s} = (W^{(2)})^T \dfrac {\delta L} {\delta o^a}\]

$\dfrac {\delta L} {\delta o^a}$ is $m \times 1$ and $(W^{(2)})^T$ is
$d_h \times m$

    \subsection{Q1 f)}\label{q1-f}

    \textbf{First show that the derivative of $tahn(z) = 1 - tanh^2(z)$}

You can see the derivation here:
http://math.stackexchange.com/questions/741050/hyperbolic-functions-derivative-of-tanh-x
.

It's not really worth the copying.

    So the derivative we are looking for is equal to:
\[\dfrac {\delta L} {\delta h^a_j} = \dfrac {\delta L} {\delta h^s_j} \dfrac {\delta h^s_j} {\delta h^a_j}\]

which is therefore equal to

\[\dfrac {\delta L} {\delta h^a_j} = \dfrac {\delta L} {\delta h^s_j} (1 - tanh^2(h^a_j))\]

$\dfrac {\delta L} {\delta h^s_j}$ is $d_h \times 1$

    \subsection{Q1 g)}\label{q1-g}

    \[\dfrac {\delta L} {\delta W^{(1)}} = \dfrac {\delta L} {\delta h^a} \dfrac {\delta h^a} {\delta W^{(1)}}\]

\[\dfrac {\delta h^a} {\delta W^{(1)}} = \dfrac {\delta L} {\delta h^a}  x^T\]

and

\[\dfrac {\delta h^a} {\delta b^{(1)}} = \dfrac {\delta L} {\delta h^a}\]

since $h^a = W^{(1)}x + b^{(1)}$

    \subsection{Q1 h)}\label{q1-h}

    $L_{mod} = \alpha W^{(1)} + \alpha W^{(2)} + \beta (W^{(1)})^2 + \beta (W^{(2)})^2 +L(x, t)$

$\dfrac {\delta  L} {\delta W^{(2)}} = \dfrac {\delta L} {\delta o^a} (h^s)^T +\alpha + 2\beta W^{(1)}  $

$\dfrac {\delta L} {\delta W^{(2)}} =  \dfrac {\delta L} {\delta o^a} (h^s)^T +\alpha + 2\beta W^{(2)}  $

    \subsection{Q1 i)}\label{q1-i}

    $\dfrac {\delta h^s} {\delta h^a} = \begin{cases} 1, & \text{if }h^a\geq\text{ 0} \\ 0, & \text{ otherwise} \end{cases}$

therefore, only the derivatives depending on this term are affected:
that is $\dfrac {\delta L} {\delta W^{(1)}}$ and
$\dfrac {\delta L} {\delta b^{(1)}}$


    % Add a bibliography block to the postdoc
    
    
    
    \end{document}
